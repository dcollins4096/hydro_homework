
   \pts{4} Derive the 2/3 law, $v_\ell^2 = \beta (\epsilon \ell)^{2/3}$
      (like eqn. 8.22) from basic physical considerations.  Make sure you motivate
      everything you say.  
      %\emph{(Try it first.  A hint for what I'm looking for
      %is on the next page.)}
    

\solution{
    \emph{Solution:}

    Within the inertial subrange, energy is not dissipated or injected.  Thus it
    can only move from one scale to another (and one position in space to
    another) but we're doing an ensemble average).  This transfer rate must
    equal the input rate, $\epsilon$ (otherwise there would be a pile up).
    $\epsilon$ can thus only depend on the size scale and the energy at that
    scale.  By either dimensional analysis, or by assuming that the timescale
    for transfer at a certain scale is $\ell/v_\ell$, the time for an eddy to
    cross itself, one can find that
    $$\epsilon \approx v_\ell^2/t = v_\ell^2/(\ell/v_\ell) = v_\ell^3/\ell$$
    Thus,
    $$v_\ell = C \epsilon^{1/3} \ell^{1/3}
    v_\ell^2 = \beta \epsilon^{2/3} \ell^{2/3}.$$
}
