
   \pts{4} Using what you got in the previous problem (which should look
    like $\dDbD{\omega}{t} = \omega \cdot \nabla \vvec$), describe why vortex
    tube stretching yields smaller structures.  (Think about what the operator
    $\omega \cdot \nabla$ means, and the conservation law for $A \omega$)

\solution{
     \emph{Solution:}
 
     The term $(\omega \cdot \nabla) \vvec $ represents gradients of each of the
     components of $\vvec$ along the line defined by $\omega$.  Since there are
     more ways for two vectors to be moving away from each other, the gradient is
     likely to be positive, thus $\dDbD{\omega}{t} > 0$ on average.  Thus,
     $\dDbD{A} < 0 $, typically
}
