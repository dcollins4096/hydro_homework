
     The sun is a problem.  At interstellar densities, 100 \percc, the gas that forms
    the sun is roughly rotating with the galactic rotation period, 250 Myr.
    It's present rotation period is 25 days.  Interstellar magnetic field is
    about $10^{-6}\rm{G}$.  It's present mean field is about 1 G.  
    \begin{enumerate}
    \item \pts{3} Assuming angular momentum conservation, what should the solar
    rotation period be?

   \item \pts{3} Assuming flux freezing, what should the solar magnetic field
   be?
   \end{enumerate}

\solution{
     \emph{Solution}  
 
     For 1 $\msun$ at $100 \percc$, that gives $\left(\frac{1 \msun}{m_{p} 100
     \percc}\right)^{1/3} = 10^{18}\rm{cm} = 2 \times 10^7 R_{\odot}$
 
     For angular momentum conservation, $M r^2/T$ is conserved, so
     \begin{align}
         \left(r_2/r_1\right)^2 T_1 &= T_2\nn\\
         10^{-14} 250 \rm{Myr} &= 19 \rm{s} << 25 \rm{days}\nn
     \end{align}
     So where did the angular momentum go?  (This is an open question, but
     much of it goes into binary systems.)
 
     Similarly, 
     \begin{align}
         B_1 (r_1/r_2)^2 = B_2\nn\\
         10^{-6} \rm{G} * 10^{14} = 10^{8} \rm{G}\nn
     \end{align}
}
