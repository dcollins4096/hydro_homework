
    Hydrostatics
       \begin{enumerate}
           \item \pts{5} Derive Equation 14.23 from 14.5.  State all the
               assumptions made. Don't forget that the unit vectors have spatial
               derivatives:
               $\dbd{\hat{\theta}}{\theta} = -\hat{r}$ and
               $\dbd{\hat{r}}{\theta}=\hat{\theta}$
            \item \pts{3} What's the significance of the last term,
                $\frac{B_\theta^2}{4 \pi r}$?
            \item \pts{3} Given $\nabla\times B=\frac{4 \pi}{c} j$, assume
                $B_z=0$, derive
                (14.24) and solve it for $B_\theta$, assuming $j$ is constant.
            \item \pts{3} Plug that back into (14.23) and solve for $p(r)$.
            \item \pts{3} Given Alfven's theorem, describe why a radial pinching
                is unstable.
       \end{enumerate}

\solution{
       \emph{Solution}

       A partial solution:

       \begin{enumerate}
           \item Set things to zero.  Do the calculus.  
           \item $B\cdot\nabla B$ is the tension:  it's the change in the field
               direction as it moves around the circle,
               $\dbd{\hat{\theta}}{\theta}=-\hat{r}$
           \item Plug in and integrate.
           \item Integrate again.
           \item Given a patch of fluid centered at $r=0$, as it contracts the
               area decreases so the field should increase.  This increases the
               inwardly-directed tension force.
       \end{enumerate}
}
