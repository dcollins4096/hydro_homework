
 Rayleigh-Taylor:  
   \begin{enumerate}

   \item \pts{3} Starting with the Rayleigh-Taylor instability configuration (
   $U=\p{U}=0$ and $\rho<\p{\rho}$),  we
     can define a growth rate $\gamma$ such that $\xi_1= e^{\gamma t} e^{i k x}$.  Write down an
     expression for $\gamma$.  
   \item \pts{2} How does $\gamma$ depend on the size of a structure? 
   \end{enumerate}

\solution{
   \emph{Solution}

   \begin{align}
       \frac{\omega}{k} &= \pm\sqrt{ \frac{g (\rho-\rho^\prime)}{k (\rho+\rho^\prime}}\nn\\
       \Gamma &= \pm k \sqrt{ \frac{g |(\rho-\rho^\prime)|}{k
       (\rho+\rho^\prime}}\nn\\
       \Gamma &\sim \sqrt{k}\nn\\
       &\sim1/\sqrt{L}\nn
   \end{align}
   so small structures will grow faster.  Thus, these modes will become
   \emph{nonlinear} first
}
