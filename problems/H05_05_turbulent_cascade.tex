
    \textbf{The turbulent cascade.}  Let's \emph{qualitatively} discuss the turbulent cascade.
    \begin{enumerate}
        \item \pts{2} I stir a pot of water with motions on some ``injection
            scale'', $L_{i}$.
            Does the energy cascade to smaller or larger scales?
      \item \pts{6} Discuss the three regimes of the cascade.  What physical
        process is dominant in each?
      \item \pts{3} Sketch the behavior of energy as a function of spatial size $\ell$.  Show
        all three of the primary regimes.
      \item \pts{3} Sketch the behavior of energy as a function of $k$.  Show
        all three of the primary regimes.
    \end{enumerate}

\solution{
  \emph{Solution:}
 
  a) \textbf{Energy} cascades to \textbf{small scale.}  
   
   c) The three regimes are the
   \textbf{Driving or integral} scale where energy is injected by some
   unspecified \emph{external force}, and fluid instabilities dominate; the
   \textbf{inertial subrange} where energy is transmitted by the \emph{convective}
   terms; and the \textbf{dissipation range} where energy is transfered to
   heat by \emph{viscosity}.  
 
    d) should look like the figure in the text. \red{NOTE that $\ell$ and $k$ should
    both increase to the right: that means that the side with the dissipation is
    different, and the slopes are opposite.}
}
