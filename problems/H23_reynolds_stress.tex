
  \pts{12} Let's derive the Reynold's Stress.       
     \begin{enumerate}
         \item Write down Navier Stokes for some $\bar{v}$ and $\bar{P}$.  This
             is our background solution that we know.   
         \item Now expand with $v=\bar{v}+\p{v}$.
         \item Then average what you get.  Use the fact that $\bar{\p{q}}$ for
             any $q$.
         \item What is the additional term that you don't have in the first
             part?  What does it represent?
     \end{enumerate}

    
    
%    We'll start with the Navier Stokes equation in conservative form.  Assume
%    $F=0$ and $\rho=const$.  Then Break each of $p, v_i$ into $f= \overline{f} +
%    \p{f}$, where $f$ is the full quantity, $\overline{f}$ is the ensemble
%    average, and $\p{f}$ is the fluctuating term.  This is called
%    \emph{Reynold's Decomposition}.   So far it looks like the perturbation
%    stuff we did at the beginnning of the semester.  This time $\p{f}$ is \emph{not small} for
%    pressure and velocity, so we \emph{can't throw anything out based on its
%    small-ness} and the mean velocity is non-zero.

%    %\red{Next time make the average of the pair of fluctuations explicit}
%
%    Ok the actual steps:
%
%
%    \begin{enumerate}
%      \item \pts{2} Show that $\overline{\overline{f}}=\overline{f}$ (the average of the average
%        quantity).  
%      \item \pts{2} Using what you just did, show that $\overline{\p{f}}=0$.
%      \item \pts{2} Insert the Reynold's decomposition into the time derivative,
%        $\dbd{\rho v_j}{t}$, expand all terms, then take the average of each
%        term.  What are you left with?
%      \item \pts{3} Do the same with the pressure and viscosity terms.
%      \item \pts{3} And finally the nonlinear term, 
%        $$\dbd{\rho v_i v_j}{x_j}$$.  Decompose the velocity into the mean and
%        fluctuating parts, expand each term, average each term.  What is different about this term and the
%        previous two?
%      \item \pts{2}  Put all the parts together, write down the evolution
%        equation for the mean velocity.  Relative to the full N.S. equation, there's a new term, the \emph{Reynold's
%        Stress}.  Describe what it's doing to the mean flow.
%    \end{enumerate}

\solution{
     \emph{Solution}
 
     This is done in the text.  Email me if you need any clarification.
}
