\pts{4} Berger's equation is Euler's equation without the pressure.  
\begin{align}
	\dbd{u}{t} + u \dbd{u}{x}=0\label{bergers}
\end{align}
where $u$ is velocity (and not constant).
Use method
of charachteristics to show that, if
\begin{align}
u(x, t=0) = -m x +b,
\end{align}
where $m$ and $b$ are constants, then
\begin{align}
u(x,t) = \frac{ b-mx}{1-mt}.
\end{align}
At what time does $u=\infty$?

\solution{\emph{Solution}

The equation \ref{bergers} is another way of saying
\begin{align}
	\ddbd{u}{t}&=0\\
	\ddbd{x}{t}&=u
\end{align}
so we have converted one PDE into two ODEs. The first one says
\begin{align}
	u(x,t) &= const\\
	&= u(x_0, t_0)
\end{align}
The second one says that
\begin{align}
	x(t) = u t + x_0,
\end{align}
or
\begin{align}
	x_0 = x(t) - u t.
\end{align}
If we start with
\begin{align}
	u(x_0, t_0) = -m x_0 + b
\end{align}
then we can plug in $x_0$ and get
\begin{align}
	u(x,t) &= -m (x - u t) + b\\
	u &= \frac{ b - m x}{1-mt}
\end{align}
and we see that this blows up when $t=1/m$.

}%solution
