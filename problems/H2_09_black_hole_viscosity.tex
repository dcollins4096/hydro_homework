
      \item The Reynolds number of a Keplerian disk (atomic hydrogen) around a
        black hole with a mass of 1 \msun:
        T = $10^4 K$, $n=10^{16} \percc$ with thickness $10^{10}$ cm.  Use the
        same viscosity formulation as the molecular cloud problem.

\solution{
         \emph{Solution}
 
         We find $\nu=10^3 \cmcmpers$ from plugging things in.  We can estimate v
         from
         $$
         v^2 = \rm{G\ M}/R
         $$
         where we know G and M, but not $R$.  For the scale in the problem, we
         use $L=h=10^{10} \rm{cm}$.   $R$, thus $v$, varies across the disk.
         Leaving it as a free parameter, we find
         $$
         Re  = h \sqrt{\rm{G}{M}/R}/\nu = 10^{19} (\frac{R}{\rm{cm}})^{-1/2}.
         $$
         To estimate $R$, one possible solution is to approximate it by the mass,
         density and thickness of the disk.  This gives $10^15$ cm, which gives
         $Re = 10^{11}$.  Any reasonable estimate should be smaller than the
         galaxy, at 25 kpc = $10^{22}$ cm, and even that gives $Re=10^8$.
